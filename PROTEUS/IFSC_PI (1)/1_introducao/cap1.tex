\chapter{Introdução}
% para fazer marcações no texto para eventuais referências, utilize o o label
% variando entre SEC FIG IMAG TAB ALG
\label{sec:introdução}

%para inicializar o primeiro parágrafo de uma seção utilize o padrão a seguir
% onde a letra na chave {A} é a letra em questão que aparecerá 
\lettrine[lines=3]{V}{}ivemos em uma era totalmente digital, onde a informação vem de forma rápida e precisa, o que antes era conhecido apenas por meio da leitura de livros e pesquisa aplicada, hoje, tem-se acesso apenas a distância de um clique. Com isso, não é de se estranhar o acúmulo de informação em meio a cidades e grandes metrópoles, sempre é possível observar letreiros, outdoors, painéis digitais, que normalmente estão sintonizados em alguma propaganda, para promover algo em questão. O dia a dia é frenético nos centros comerciais, então o investimento em tecnologias desse porte no meio dos donos de lojas e estabelecimentos tem se tornado cada vez mais frequente. "O visual é a primeira impressão que temos de um lugar, e a primeira imagem que um cliente tem do seu estabelecimento é a fachada. Por isso, investir em um bom layout para letreiros e fachadas é um diferencial para a divulgação de empresas e  a atração de novos clientes" \cite{SPletras}, tendo em mente toda essa onda dos meios de informação, o grupo decidiu por desenvolver um painel de LEDs, que permitirá o compartilhamento de qualquer tipo de texto por meio de painéis escaláveis, controlado via internet.
 
\section{Contextualização ou definição do problema}
	
Atualmente os painéis e letreiros apresentam configuração através de drivers inbutidos programavéis via computador, uma das metas do projeto é tornar essa configuração mais simples, por meio de uma interface gráfica via internet, a outra meta é tornar o painél escalável, permitindo a junção de dois ou mais painéis para a criação de um maior, sem que isso afete o texto no mesmo.


\section{Objetivos}

O trabalho visa a criação de uma placa de leds escalável com comunicação SPI para que possa haver a expansão da placa,para esse semestre elá será expandida apenas horizontalmente, mas o hardware permitirá a expansão vertical, e a mesma pode ser controlada pelo computador atráves de uma interface gráfica.
		
		\begin{itemize}
			\item Confeccionar uma placa de circuitos impressos (PCI).
			
			\item Criar a interface gráfica que permita ao usuário escolher de forma acessível o que irá aparecer na placa.


		\end{itemize}

